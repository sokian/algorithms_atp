\documentclass{article}
\usepackage[T1]{fontenc}
\usepackage[utf8]{inputenc}
\usepackage{amsmath,amssymb}
\usepackage[russian,english]{babel}



\begin{document}
\selectlanguage{russian}
\textbf{Общие замечания.}
Для начала заметим что любая команда из 2-ух и менее человек подходит под ограничения задачи и всегда можно выбрать 1 или 2 игроков с максимальными эффективностями.
Пусть теперь наша оптимальная команда состоит из $n > 2$ игроков. Среди них есть два с минимальной эффективностью,
пусть их суммарная эфферктивность $E_min$, а их эффективности $E_1$ и $E_2$ соответственно, тогда в эту комманду можно
включить любого игрока с эффективностью лежащей в отрезке $[E_2;E_min]$.
\\
\\
\textbf{Утверждение.}
Пусть $E_1...E_n$ отсортированный по неубыванию список эффективностей игроков, тогда оптимальный ответ на задачу достигается на каком-либо
подотрезке этого списка.
\\
\textbf{Доказательство.}
Допустим что оптимальный ответ состоит из $n > 2$ игроков (случай меньшего количества разобран выше). И стреди их эффективностей в отсортириванном списке имеются проруски. Пусть есть пропущенный игрок с индексом j в отсортированном списке, тогда все игроки включенные в команду имеют эффективность меньше либо равную $E_j$. Выберем вместо этих игроков игроков с индексами на 1 больше тогда суммарная эффективность стала не меньше и сумма двух минимальных не уменьшилась, т.е. условие сплоченности не нарушилось, и суммарная эфферктивность такой команды не хуже оптимальной. Таким образом му можем произвести такие сдвиги игроков и получить команду, которая является подотрезком в отсортированном списке эффективностей и суммарная их эффективность не уменьшится.
\\
\\
\\
В итоге задача свелась к следующей - нахождение подотрезка максимальной суммы в отсортированном массиве при условии, что максимальный элемент не больше суммы двух минимальных.
\\
\textbf{Алгоритм.}
Пусть длина такого отрезка $>=2$ (один элемент очевидный случай). Переберем начало такого отрезка $i$, пусть $E = E_i + E_{i + 1}$, найдем самого последнего игрока, у которого эффективность меньше либо равна $E$ (юудем искать его бинарным поиском за $O(log(n))$), пусть его индекс $last$. Тогда самый длинный отрезок начинающийся с $i$ это $[i;last]$. И мы можем включить в оптимальную команду всех этих игроков. Их суммарную эффективность будем вычислять используя предварительно подсчитанный массив частичных сумм за O(1). И каждый раз обновлять оптимальный ответ текущим.
\\
\textbf{Сложность.}
Сортировка элементов $O(nlog(n))$. Далее для каждого начала отрезка делаем бинарный поиск правого конца этого отрезка и за $O(1)$ находим сумму элементов этого отрезка. Итого вторая часть тоже $O(nlog(n))$.
\\
\textbf{Память.}
Расход памяти $O(n)$ (хотя это может зависеть от используемого алгоритма сортировки).

\end{document}